\addcontentsline{toc}{chapter}{Remerciements}
\begin{center}
\textbf{\large Remerciements}
\end{center}

Un grand merci \`a mes colocataires des trois derni\`eres ann\'ees, J\'er\^ome,
Claire, Estelle puis plus tard No\"el ainsi qu'\`a mes ami(e)s en particulier
Awa et Bianka, qui ont \'et\'e aux premi\`eres loges pour partager mes joies et
frustrations mais surtout pour avoir offert un environnement qui m'a permis de
d\'ecrocher quand j'en avais besoin.

Merci \`a Paul (Khuong) et Paul (Raymond-Robichaud) pour les discussions
stimulantes, \'Etienne et David pour le partage de leur exp\'erience du milieu
acad\'emique, ce qui a bris\'e mon sentiment d'isolement aux moments o\`u j'avais
l'impression d'\^etre seul face \`a certaines difficult\'es.

Merci \'egalement \`a tous ceux qui sont pass\'es par le lab pendant mon s\'ejour, en
particulier Beno\^it, pour la co-cr\'eation de la version la plus \'epique d'un
Pacman 3D moustachu \`a laquelle j'aie particip\'e, Eric, Vincent et Benjamin pour
avoir d\'emontr\'e un int\'er\^et pour mes nombreuses "d\'emos" de versions interm\'ediaires
du syst\`eme ainsi que Maxime pour m'avoir montr\'e un calibre de programmation que
je n'avais pas connu jusqu'\`a mon entr\'ee au lab.

Merci au personnel administratif du d\'epartement d'informatique
et de recherche op\'erationnelle, en particulier Mariette Paradis, pour les
nombreux accomodements et les judicieux conseils prodigu\'es pour la navigation
au travers des requis administratifs inh\'erents \`a la poursuite d'\'etudes gradu\'ees
dans un contexte institutionnel. Cela a grandement simplifi\'e ma vie.

Merci \`a Bruno Dufour, qui aurait m\'erit\'e d'\^etre officiellement co-directeur (ou
plus!) vu son niveau d'implication dans mon projet de ma\^itrise.
L'identification du probl\`eme d'instrumentation dynamique efficace est une
cons\'equence directe de nos discussions. L'instrumentation dynamique a fourni un
usage pratique \`a une solution qui \'etait en recherche d'un probl\`eme. Tu es
\'egalement le premier \`a avoir pleinement reconnu l'int\'er\^et acad\'emique de mon
travail.

Merci \`a Marc Feeley, mon directeur, d'avoir servi d'exemple sp\'ecifiquement
par son soucis du d\'etail et du mot juste ainsi que ses capacit\'es techniques
remarquables.  Bien que \,ca n'ait pas toujours \'et\'e intentionnel, merci
\'egalement de m'avoir sorti de ma zone de confort. Mes limites personnelles
ont \'et\'e dans certains cas repouss\'ees et dans d'autres cas raffermies. 
