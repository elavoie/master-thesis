\chapter{Performance}
\funnyquote{There is a deep difference between what we do and what mathematicians
do. The "abstractions" we manipulate are not, in point of fact, abstract. They
are backed by real pieces of code, running on real machines, consuming real
energy and taking up real space. To attempt to completely ignore the underlying
implementation is like trying to completely ignore the laws of physics; it may
be tempting but it won't get us very far.}
{Gregor Kiczales~\cite{Kiczales92towardsa}}

\label{chap:Performance}

The last section has shown how the additional flexibility provided by the
system could be used to instrument the VM and modify the semantics of some of
its operations. We now evaluate the actual run-time and memory-usage overhead
incurred by the additional flexibility.

The next section explains the methodology used to evaluate the performance of
the system. We then present the baseline performance and memory overhead
obtained on different systems for common benchmarks. We then see the impact of
a chosen instrumentation on run-time performance and memory usage and we
conclude with an interpretation of the results. 

\section{Methodology}

We compare three different systems:
\begin{itemize}
    \item Photon running over V8 (V8's full optimizations)
    \item V8 (full optimizations)
    \item SpiderMonkey interpreter (JIT disabled)
\end{itemize}

V8 was chosen to host Photon because preliminary tests showed the system to be
faster on it and V8 is known to be one of the fastest JS VMs. The additional
speed is attributed to the ability of the runtime to do function inlining,
on-stack-replacement and the presence of fast garbage collectors. Other VMs are
catching up on features and the current focus seems to be on method-based
Just-In-Time compilers so we anticipate that in the near future they could
probably be used interchangeably. We compare Photon to V8 to quantify the
performance cost incurred by our approach.  We finally compare to a popular
state-of-the-art interpreter, SpiderMonkey used in Firefox, to argue that our
approach can be used wherever a manual instrumentation of that interpreter
could have been performed. 

In choosing the systems to compare, we eliminated a few current alternatives.
We assume that when faced with the task of instrumenting production code to
obtain run-time data, manually instrumenting a JIT-compiler would be deemed too
complex to be cost-effective in terms of development time. At the time of
writing, JavaScriptCore's low-level interpreter became available and replaced
the original interpreter that was instrumented by Richards and
al.~\cite{behavior_js} in WebKit.  We argue that the only real instrumentation
alternative right now would be SpiderMonkey's interpreter because the
JavaScriptCore low-level interpreter is implemented in an assembly dialect to
obtain performance gains. As this new interpreter matures, we speculate that
its complexity will increase, negating most of the simplicity usually
attributed to interpreters.~\footnote{Should the reader still consider the
alternative of instrumenting JavaScriptCore's low-level interpreter instead,
she should know that in our tests, on V8's benchmarks, JavaScriptCore low-level
interpreter was roughly three times faster than SpiderMonkey's interpreter. How
much of those speed gains would remain in presence of instrumentation is
unknown.} Finally, we do not show performance results for Narcissus, Mozilla's
JavaScript in JavaScript interpreter, because the latest version would not run
either SunSpider or V8 benchmarks.

To assess performance, we use the SunSpider and V8 benchmark suites, since they
became the \textit{de facto} standard to compare JS VM performance. They both
come with their own different evaluation methodology. We used the original
methodology of each benchmark suite to make our results comparable to other
published results for other systems. All the benchmarks were run unmodified.
However, the date benchmarks from SunSpider were omitted because they relied on
using \kw{eval} to access local variables, which is not supported by our
current compiler.  Since Photon's local environment uses the underlying VM's
local environment (it is not reified), we believe that supporting this feature
in the future will not significantly affect the performance results presented
here.

We focus on two metrics, running time and the maximum heap size to respectively
measure run-time performance and memory usage. Running time is measured either
directly, in the case of SunSpider's benchmarks or indirectly through a score,
in the case of V8's benchmarks. For V8's benchmarks, higher scores mean a
faster system (less execution time). For SunSpider benchmarks, its the
opposite, lower scores mean a faster system.  Memory usage is used to estimate
the overhead of Photon compared to V8, but it was not measured for the
SunSpider interpreter because the information was hard to obtain.

We present results in two different groups, the baseline performance and the
instrumented performance.  The baseline performance is used to measure the
minimal overhead of the approach since it determines its viability, regardless
of other characteristics.  The instrumented performance is used to measure the
impact of instrumentation on the baseline performance. It is common knowledge
that instrumenting an interpreter has little impact over its overall
performance (this was verified by Richards and al. when they instrumented
JavaScriptCore~\cite{behavior_js}).  However, our approach is more sensitive to
instrumentation. We therefore quantify its impact. 

The instrumentation chosen counts the number of occurrences at run time of each
reified object-model operation, in the same spirit as the example given in
section~\ref{sec:ObjectModelInstrumentation}. The original operation is inlined
in the replacement method instead of being called. The actual code is given in
section~\ref{sec:InstrumentedPerformance} to show that the complexity is almost
the same although we eliminated a function call.  We chose this particular
instrumentation because it is simple, it covers important object model
operations and it was actually used to gather information about JS (it can be
used to reproduce the object read, write or delete proportion figure
from~\cite{behavior_js}).

All results were obtained on a MacBook Pro running OS X version 10.7.5 with a
2.2 GHz Intel Core i7 processor and 8 GB of 1333 MHz DDR3 Ram. We used V8
revision 12808 and SpiderMonkey version 1.8.5+ 2011-04-16. The results are
intended to give an approximation of performance for the approach. Therefore,
it was not deemed necessary to provide confidence intervals and averages of
multiple runs, given that each test harness already did so. Although some
variations between runs were noticed, they were sufficiently stable to not
affect our argument.

For conciseness, abbreviations are used in the tables. They are listed in order
of appearance:
\begin{itemize}
    \item Pn: Photon
    \item SM: SpiderMonkey
    \item V8: V8
    \item Pn-spl: Photon with our simple instrumentation
    \item Pn-fast: Photon with an equivalent instrumentation optimized for speed
\end{itemize}

\section{Baseline performance}

\subsection{Running times}

The baseline performance for V8 benchmarks is shown in
Table~\ref{tb:BaselinePerformanceV8}. It is pretty good compared to the
SpiderMonkey interpreter for an overall score within a factor of 2. On three of
the eight benchmarks Photon is faster and in two cases by almost a factor of
two. In other cases, the SpiderMonkey interpreter is between 2 and 3.5 times
faster, except for the Splay benchmark where it is 13 times faster. 

This last case seems to be a pathological case for the basic optimizations
performed on property access, assignment and update. As will be seen in
Table~\ref{tb:InstrumentedPerformanceV8}, for this particular benchmark, the
instrumented version of Photon is five times faster than the non-instrumented
version. This can be explained by the fact that the instrumented version
\textit{removes} the optimizations attempted. However, removing the same
optimizations for all benchmarks gives an overall score 30\% inferior with some
benchmarks almost four times slower, except for Splay. Should we take the
fastest score obtained without optimizations for Splay, Photon would be two
times slower which is comparable to the other results.

\begin{table}[htb]
\caption{Baseline performance on V8 benchmarks}
\centering
    \input{/Users/erick/Recherche/photon-js/results/baseline/v8/time/table.tex}
\label{tb:BaselinePerformanceV8}
\end{table}

The baseline performance for SunSpider benchmarks, shown in
Table~\ref{tb:BaselinePerformanceSunSpider} gives an overall
score 1.6 times the SpiderMonkey interpreter. In this case, we use the
geometric mean of the ratios because some pathological cases might give a wrong
picture of the actual performance of the system if we were to simply use the
total of all the running times to compute the overall ratios.  10 of 24
benchmarks are faster on Photon than on SpiderMonkey, with some as much as 5
times faster. This can be attributed to faster basic operations as compiled by
the V8 JIT Compiler compared to the interpreted operations of SpiderMonkey. The
slowest times are obtained on all string manipulation benchmarks such as
crypto-* and string-*. One notable slow case is the string-tagcloud benchmark,
which uses \kw{eval} for JSON parsing, that calls into our compiler.  The
bottleneck seems to be the parsing stage of our OMeta-based compiler~\cite{Warth:2007}.
Experiments with other compilation techniques, still using JavaScript as an
implementation language, were made in our lab with performance results than
would make the compilation time negligible. For the rest, it seems that
auto-boxing of strings and the wrapping of functions passed to the
standard-library functions would be responsible for most of the slowdown.


\begin{table}[htb]
\caption{Baseline performance on SunSpider benchmarks}
\centering
    \input{/Users/erick/Recherche/photon-js/results/baseline/sunspider/time/table.tex}
\label{tb:BaselinePerformanceSunSpider}
\end{table}

\subsection{Memory usage}
The heap size is measured from garbage collection traces. When no garbage
collection has been done, a result of '--' is given, which appears in some of V8
benchmark runs.

Memory usage for both benchmark suite in Table~\ref{tb:BaselineMemoryV8} and
Table~\ref{tb:BaselineMemorySunSpider} look good given that every run-time
object has an associated proxy with most cases using less than a factor of two
in memory and two worst cases of 6.5 and 10. The high memory usage in
EarlyBoyer might be explained by the memory overhead of every inline cache
site having an associated array. In the case of string-unpack-code, the
overhead might be explained by string and regexp proxies.

\begin{table}[htb]
\caption{Baseline memory usage on V8 benchmarks}
\centering
    \input{/Users/erick/Recherche/photon-js/results/baseline/v8/memory/table.tex}
\label{tb:BaselineMemoryV8}
\end{table}

\begin{table}[htb]
\caption{Baseline memory usage on SunSpider benchmarks}
\centering
    \input{/Users/erick/Recherche/photon-js/results/baseline/sunspider/memory/table.tex}
\label{tb:BaselineMemorySunSpider}
\end{table}

\newpage
\section{Instrumented performance}
\label{sec:InstrumentedPerformance}

The instrumentation used for performance evaluation is a variation of the
instrumentation given in section~\ref{sec:ObjectModelInstrumentation}, that
inlines the object model operation instead of wrapping it in another function. 
The simple version does not exploit the memoization protocol and
corresponds to the straight-forward implementation:

\jsfile{/Users/erick/Recherche/photon-js/perf/instrumentation.js}

The fast version inlines the instrumentation operation inside the optimized
version of object operations for speed:

\newpage
\jsfile{listings/object-model-instrumentation-fast-1.js}

\jsfile{listings/object-model-instrumentation-fast-2.js}

\jsfile{listings/object-model-instrumentation-fast-3.js}

The first version is intended to measure what performance can be expected from a
quickly developed instrumentation while the second one is intended to measure
the performance impact of the instrumentation operations alone. This is
therefore a low-barrier high-ceiling example and illustrates the flexibility
that can be gained when the choice of aiming for performance is left to users
of the system.

\subsection{Running times}

Table~\ref{tb:InstrumentedPerformanceV8} and
Table~\ref{tb:InstrumentedPerformanceSunSpider} show the impact of
instrumentation on the baseline performance. In both cases, the fast version has
negligible impact with some results slightly faster, mostly due to natural
variation of results.  However, the simple version can be as much as 30 times
slower on some benchmarks.

\begin{table}[htb]
\caption{Instrumented performance on V8 benchmarks}
\centering
    \input{/Users/erick/Recherche/photon-js/results/instrumented/v8/time/table.tex}
\label{tb:InstrumentedPerformanceV8}
\end{table}


\begin{table}[htb]
\caption{Instrumented performance on SunSpider benchmarks}
\centering
    \input{/Users/erick/Recherche/photon-js/results/instrumented/sunspider/time/table.tex}
\label{tb:InstrumentedPerformanceSunSpider}
\end{table}

\subsection{Memory usage}

Table~\ref{tb:InstrumentedMemoryV8} and
Table~\ref{tb:InstrumentedMemorySunSpider} show that on V8 and SunSpider
benchmarks, the simple and fast instrumentations have no impact on memory
usage, with the exception of the Splay benchmark in the simple case.  This might
be caused by our tracking strategy that has an entry for every cached regular
method call. This might be removed not tracking certain method names such as
\kw{__get__}, \kw{__set__} and \kw{__delete__} and conservatively invalidating
all caches when they are redefined.

\begin{table}[htb]
\caption{Instrumented memory usage on V8 benchmarks}
\centering
    \input{/Users/erick/Recherche/photon-js/results/instrumented/v8/memory/table.tex}
\label{tb:InstrumentedMemoryV8}
\end{table}

\begin{table}[htb]
\caption{Instrumented memory usage on SunSpider benchmarks}
\centering
    \input{/Users/erick/Recherche/photon-js/results/instrumented/sunspider/memory/table.tex}
\label{tb:InstrumentedMemorySunSpider}
\end{table}


\section{Interpretation}

%Our current results show that the baseline performance (without
%instrumentation) is around two times slower than the SpiderMonkey interpreter
%both on V8 and SunSpider benchmarks and the baseline memory usage is in most
%cases, around twice that of V8. A basic instrumentation slows the system by an
%additional factor of three but has almost no effect on memory usage, while the
%same instrumentation optimized for speed can have negligible impact both on
%run-time performance and memory usage compared to the baseline.

A conclusion can be made that our current optimizations are not sufficiently
stable across benchmark results and a better predictor of whether to optimize
or not would give more even results, especially for V8's Splay benchmark. We
believe this can be done within the framework presented. Further, the memory
overhead seems acceptable for instrumentation tasks given our multi-gigabytes
of RAM in today's machines. Instrumentation can have a negligible impact both
on performance and memory usage, as long as the operations instrumented are
optimized enough. Otherwise, the choice of trading performance for a faster
development time can still be made with an impact on performance that can be
reasonable in most cases.

The reader should keep in mind that our focus has been first and foremost on
bringing flexibility to JS and as explained in the design chapter, some known
optimizations can still be applied for further possible gains that are yet to
be quantified. We therefore argue that the performance obtainable by our
approach is \textit{at least} what we present here but the upper bound is still
to be found and will be the object of further research. The conclusion presents
promising ideas for improving both the baseline performance and instrumented
performance.

