\chapter{Previous Work}
\emph{``Those who cannot remember the past are condemned to repeat it.''} 

--- George Santayana, $1863-1952$ \\

Turning our attention to previous work can serve a number of purposes such as
establishing the origin of fundamental concepts behind an idea,
collecting learning material for the interested reader and demonstrating the
originality of the work presented. Being free of the requirement of
originality, since this dissertation is a Master thesis, we will focus on the
first two.

In the course of this litterature review, it became clear that the
terminology around flexible computer systems is both varied and ill-defined.
Synonyms to flexibility abound: \textit{malleability} brings images of
amorpheous substrates, \textit{versatility} testifies for the universality of
the artifact, \textit{plasticity} suggests a solid while reconfigurable medium
and \textit{customizability} reminds of color options for mass-produced
consumers goods. While nuanced in their suggestive tones, usage of these terms
share the unfortunate characteristic of lacking a precise definition which
might serve as a basis for expectations toward the system they describe.

The consequence is that comparison between different systems built around
different organizing principles is difficult and becomes almost impossible when
the systems belong to different sub-disciplines, such as operating systems and
virtual machine construction. Asserting that a system is flexible amounts to
little more than a wish by the authors that a system will be easy for others or
themselves to use for needs beyond the documented cases.  

Untangling the issue of characterizing existing flexible systems according to
the same framework is beyond the scope of this dissertation, as it would mostly
be useful for establishing originality. However, this does not mean that we
will fall prey to a "flexibility" relativism. The four properties of
\textit{openness}, \textit{extensibility}, \textit{dynamism} and
\textit{performance} will guide our review of previous work, by grounding our
understanding of flexibility in a richer and recognized definition. Still, the
exact meaning of those terms still vary in the litterature but we shall be
careful in precising our usage.

Flexibility applies to different components of a VM. To make the matter more
manageable and to support the thesis without undue detour, this review will be
restricted to issues pertaining to the object model. 

The work described in this dissertation started as an exploration of the
possibilities offered by an object model based on the open, extensible object
model by Piumarta and Warth~\cite{Piumarta:2008}. The idea of characterizing a
system in terms of four properties was also heavily inspired by their usage of
the terms, although precise definitions were not provided.

The next section provides the historical roots of concepts belonging to each of
the four properties and the last section suggests reading material for the
reader motivated for further independent study. This distinction is motivated
by the fact that later accounts of ideas can additionaly present further
developments, changes in nomenclature and might be more approachable then the
original papers.

\section{Historical roots}

\subsection{Openness}

The meaning of open depends on what is being qualified. Perhaps the most common
usage is open \textit{source}. It refers to accessibility of the text
representation, i.e. source code, that was used to produce the system.
Accessibility to the source and development tools allows a motivated person to
modify the text representation and generate a new version of the system
incorporating the change. However, it says nothing about the nature and the
number of changes that will be required to obtain a desired behavior.

We will use open as an implementation qualifier. It therefore means that
the behavior of the system can be modified by first-class data structures. The
range of behavior modifications allowed by an implementation characterize the
degree of openness.

In chronological order, the following work contributed to key ideas behind open
implementations:
\begin{itemize}
    \item Toward relaxing assumptions in languages and their implementation,
        Mary Shaw and Wm A Wulf, 1980, Earliest account of the need to open
        some parts of the implementation
    \item Procedural Reflection in Programming Languages, Brian Smith, 1982,
        Introduces terminology to think about reflection such as Knowledge
        Representation Hypothesis, Reflection Hypothesis, Causal Connection and
        propose a sound model for reflection, the Infinite Tower of Reflection
    \item Concepts and Experiments in Computational Reflection, Pattie Maes,
        1987, Introduces five properties to characterize reflection
        architectures (separation, uniformity, completeness, consistency,
        dynamic) and meta-objects to bring reflection to object-oriented
        programming languages 
    \item Implementational Reflection in Silica, Ramanao Rao, 1991, Introduce
        the term open implementation by analogy with a reflection architecture
        but for applications and not only programming language implementations
    \item Open Implementation Analysis and Design, Chris Maeda et al., 1997,
        Propose a methodology for designing open modules
\end{itemize}

% What about Smalltalk? Meta-stability, Bootstrap?

\subsection{Extensibility}

Extensible means that a VM's components can be independently modified or
replaced, that new components can be supported without having to modify
existing components and they can be incrementally defined in terms of existing
components. Encapsulation and modularity covers a subset of that definition. In
chronological order:

\begin{itemize}
    \item Some Features of the SIMULA 67 Language, 1966, Ole-Johan Dahl, BJørn
        Myhrhaug, Kristen Nygaard, Introduces the concepts of class-based
        inheritance and objects
    \item On the criteria to be used in Decomposing Systems into Modules, David
        L. Parnas, 1972, Introduces the concept of information hiding for
        modular decomposition
    \item On the discipline of Programming, 1976 and On the role of scientific
        thought, Edsger W. Dijkstra, Introduces the term separation of concerns
        which replaced information hiding, although it meant pretty much the
        same thing
    \item Using prototypical objects to implement shared behavior in
        object-oriented systems, 1986, Henri Lieberman, Compared prototype-based
        and class-based inheritance, showed that the prototype version was
        superior because it could implement the class version but the reverse
        was not true
\end{itemize}


\subsection{Dynamism}

Dynamism has seen an increasing usage through the popularization of dynamic
languages. Again, the actual meaning of dynamism in dynamic language is
fuzzy. The common thread however is the presence of \textit{late-binding} of
some aspects of the programming language, i.e. deferring until run time the
actual computation required to implement a given functionality.

Limiting our review to dynamic aspects of object models yields the following
fundational work.  Notice that some article mentioned previously appear again
since the concepts were pioneered in the context of dynamic languages:
\begin{itemize}
    \item Recursive functions of symbolic expressions and their computation by
        machine, Part I, John MacCarthy, 1960, Earliest account of dynamic (strong) typing 
    \item Actor Induction and Meta-evaluation, 1973, Carl Hewitt et al. Actor
        asynchronous message-passing as a primitive operation (polymorphism)
    \item Smalltalk, 1972-1976, Synchronous message-passing as a primitive
        operation (polymorphism)
    \item Procedural Reflexion in Programming Languages, 1982, Brian Cantwell
        Smith, Introduced behavioral reflection, reflection on the run time behavior of programs, although the term procedural reflection is used
    \item The LOOPS Manual, 1983, CommonLoops: merging Lisp and object-oriented
        programming, 1986, Daniel G. Bobrow, Introduced multiple dispatch (polymorphism) as a generalization of message sending (single-dispatch)
    \item Using prototypical objects to implement shared behavior in
        object-oriented systems, 1986, Henri Lieberman Introduced the delegation mechanism
        occuring at run time
\end{itemize}

\subsection{Performance}
In terms of software-based optimizations to remove much of the overhead of the dynamism the following work pioneered many optimizations:
\begin{itemize}
    \item Efficient implementation of the Smalltalk-80 system, 1984, L Peter
        Deutsch and Allan M. Schiffman, Introduces the inline-cache technique
    \item An efficient implementation of SELF a dynamically-typed
        object-oriented language based on prototypes, 1989, Introduces the maps
        implementation concept to allow inline caches to work for
        prototype-based systems 
    \item Efficient Method Dispatch in PCL, 1990, Gregor Kiczales and Luis
        Rodriguez, Introduces optimizations for multiple-dispatch functions
\end{itemize}


\section{Learning Material}

\subsection{Litterature}
\begin{itemize}
    \item The Art of the Metaobject Protocol, 1995, Gregor Kiczales et al. Uses
        the Common Lisp Object System for explaining how to open a
        language implementation through a meta-object protocol
    \item Open Implementations and Metaobject Protocols, Gregor Kiczales and
        Adreas Paepcke, 1996, Tutorial book on opening an object model for a
        programming language
    \item Actor Model of Computation: Scalable Robust Information Systems,
        2010, Carl Hewitt, Introduce the Actor model at a high-level, relate to
        existing work and provide pointers to existing litterature on Actors
\end{itemize}

\subsection{Instrumenting JavaScript Virtual Machines}
\begin{itemize}
    \item JSMeter
    \item Empirical evaluation of JavaScript Behavior
\end{itemize}


\subsection{Systems}
\begin{itemize}
    \item lisp/scheme
    \item smalltalk
%    \item forth
    \item slate
    \item maru
    \item lua
%    \item Maxine, OVM, Moxie, JikesRVM
%    \item VMKit, 
    \item Google Tracer compiler
    \item Mozilla Shamway
    \item JSBench
\end{itemize}
