
\addcontentsline{toc}{chapter}{R\'esum\'e}
\begin{center}
\textbf{\large R\'esum\'e}
\end{center}


\vspace{1cm}

L'observation de l'ex\'ecution d'applications JavaScript est habituellement
r\'ealis\'ee en instrumentant une machine virtuelle (MV) industrielle ou en
effectuant une traduction source-\`a-source \textit{ad hoc} et complexe. Ce
m\'emoire pr\'esente une alternative bas\'ee sur la superposition de machines
virtuelles. Notre approche consiste \`a faire une traduction source-\`a-source
d'un programme pendant son ex\'ecution pour exposer ses op\'erations de bas
niveau au travers d'un \emph{mod\`ele objet} flexible. Ces op\'erations de bas
niveau peuvent ensuite \^etre red\'efinies pendant l'ex\'ecution pour pouvoir
en faire l'observation.  Pour limiter la p\'enalit\'e en performance
introduite, notre approche exploite les op\'erations rapides originales de la
MV sous-jacente, lorsque cela est possible, et applique les techniques de
compilation \`a-la-vol\'ee dans la MV superpos\'ee.  Notre impl\'ementation,
Photon, est en moyenne 19\% plus rapide qu'un interpr\`ete moderne, et entre
\factor{19} et \factor{56} plus lente en moyenne que les compilateurs
\`a-la-vol\'ee utilis\'es dans les navigateurs web populaires. Ce m\'emoire
montre donc que la superposition de machines virtuelles est une technique
alternative comp\'etitive \`a la modification d'un interpr\`ete moderne pour
JavaScript lorsqu'appliqu\'e \`a l'observation \`a l'ex\'ecution des
op\'erations sur les objets et des appels de fonction.

Mots-cl\'es: M\'eta-circularit\'e, Instrumentation, Dynamisme, Mod\`ele Objet,
Flexibilit\'e, Performance, Machine Virtuelle, JavaScript  
