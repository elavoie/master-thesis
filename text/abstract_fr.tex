
\addcontentsline{toc}{chapter}{R\'esum\'e}
\begin{center}
\textbf{\large R\'esum\'e}
\end{center}


\vspace{1cm}
Les d\'eveloppements r\'ecents sur les machines virtuelles (MVs), en particulier
pour le language JavaScript, ont mis l'emphase sur la performance au d\'etriment
de la flexibilit\'e. Cela limite notre compr\'ehension du comportement dynamique
des programmes en rendant laborieuse l'instrumentation de MVs existantes. Les
approches pass\'ees consistaient \`a instrumenter manuellement un interpr\'eteur
commercial, limitant l'acquisition de donn\'ees longitudinale, \`a cause du co\^ut
\'elev\'e de maintenance de l'interpr\'eteur.
			
Ce m\'emoire montre que la flexibilit\'e peut \^etre r\'ecup\'er\'ee dans l'instrumentation
dynamique du mod\`ele objet et des appels de fonctions, \`a un niveau de
performance comp\'etitif avec un interpr\'eteur commercial r\'ecent, sans
modification de son code source. Notre approche consiste \`a ex\'ecuter une MV
m\'eta-circulaire, ciblant le langage source, sur une MV rapide. Pour \'evaluer
l'approche, nous proposons une MV pour JavaScript, nous pr\'esentons des exemples
d'instrumentation et nous comparons la performance, avec et sans
instrumentation, \`a l'interpr\'eteur SpiderMonkey et la MV bas\'ee sur un JIT de V8.
			
Nous croyons que la combinaison de simplicit\'e, de flexibilit\'e et d'efficacit\'e
de notre MV est unique. Elle est rendue possible par trois contributions:

\begin{itemize}
    \item L'unification des op\'erations r\'eifi\'ees du mod\`ele objet et des appels
        de fonctions autour d'une primitive unique de passage de message,
        compatible avec la derni\`ere version de JavaScript
    \item Une impl\'ementation efficace de la primitive de passage de message
        inspir\'ee par la m\'emoisation \textit{in situ}.
    \item Une repr\'esentation objet qui exploite les optimisations de
        m\'emoisation \textit{in situ} de la VM sous-jacente et le dynamisme du mod\'ele
        objet pour obtenir des op\'erations virtualis\'ees efficaces 
\end{itemize}

Mots-cl\'es: M\'eta-circularit\'e, Instrumentation, Dynamisme, Mod\`ele Objet,
Flexibilit\'e, Performance, Machine Virtuelle, JavaScript  
