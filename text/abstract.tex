\addcontentsline{toc}{chapter}{Abstract}
\begin{center}
\textbf{\large Abstract}
\end{center}

\vspace{1cm}

Run-time monitoring of JavaScript applications is typically achieved by
instrumenting a production virtual machine or through ad-hoc, complex
source-to-source transformations. This dissertation presents an alternative
based on \emph{virtual machine layering}. Our approach performs a dynamic
translation of the client program to expose low-level operations through a
flexible \emph{object model}. These low-level operations can then be redefined
at run-time to monitor the execution. In order to limit the incurred
performance overhead, our approach leverages fast operations from the
underlying host VM implementation whenever possible, and applies Just-In-Time
compilation (JIT) techniques within the added virtual machine layer. Our
implementation, Photon, is on average 19\% faster than a state-of-the-art
interpreter, and between \factor{19} and \factor{56} slower on average than the
commercial JIT compilers found in popular web browsers.  This dissertation
therefore shows that \emph{virtual machine layering} is a competitive
alternative approach to VM instrumentation when applied to run-time monitoring
of object operations and function calls.  
			

Keywords: Metacircularity, Instrumentation, Dynamism, Object Model,
Flexibility, Performance, Virtual Machine, JavaScript  
