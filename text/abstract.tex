\addcontentsline{toc}{chapter}{Abstract}
\begin{center}
\textbf{\large Abstract}
\end{center}

\vspace{1cm}

Recent research and development on Virtual Machines (VMs), especially for the
JavaScript language, has focused on performance, to the expense of flexibility.
Notably, it has hindered our understanding of the run-time behavior of programs
by making instrumentation of existing VMs laborious. Past approaches required
manual instrumentation of production interpreters, preventing the acquisition
of longitudinal data because of the high cost of maintaining the interpreter
up-to-date.
			
This dissertation shows that performance can be harnessed to provide flexible
run-time instrumentation of the object model and function-calling protocol at a
performance competitive with a state-of-the-art interpreter, without having to
modify the VM source code. Our approach consists in running a metacircular VM
targeting the source language, based on a message-sending object model, on top
of another fast VM. To demonstrate the approach, we provide a reference VM for
JavaScript, we show the possibility of instrumenting the object model
operations and function calls and we finally compare the performance with and
without instrumentation to the SpiderMonkey interpreter and V8 JIT-compiler
based VM.
			
We believe our combination of simplicity, flexibility and efficiency is unique.
As such, this dissertation contains three original contributions:

\begin{itemize}
    \item The unification of the reified object model operations and
        function-calling protocol around a single message-sending primitive
        while preserving compatibility with the current version of JavaScript
    \item An efficient implementation of the message-sending primitive inspired
          by inline cache optimizations
    \item An object representation exploiting the underlying VM inline caches
          and dynamic object model to provide efficient virtualized operations
\end{itemize}

Keywords: Metacircularity, Instrumentation, Dynamism, Object Model,
Flexibility, Performance, Virtual Machine, JavaScript  
