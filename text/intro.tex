\chapter{Introduction}

\emph{``Impermanent are all component things,\\
They arise and cease, that is their nature:\\
They come into being and pass away,\\
Release from them is bliss supreme.''} 

--- Mahaa-Parinibbaana Sutta \cite{1988last} \\

Computer systems are constantly modified to satisfy the evolving needs of their
users. \textit{Flexible} systems evolve faster than rigid systems by placing fewer
constraints on possible modifications, thus reducing the delay between the
birth of an idea and its actual realization in a working system. Aiming for
flexibility in the design of computer systems can make other properties, such as
performance, security and reliability, easier to obtain by facilitating
experimentation.

A \textit{virtual machine} (VM) is a program that simulates the behavior of a
computing machine.  The simulated machine might exhibit properties that do not have
physical equivalents. For example, by using a garbage collection component, it
can provide infinite allocatable memory. Programs executing on VMs inherit
their properties and constraints, therefore making them prime research objects
to provide properties to an important number of programs.

Recent research and development on VMs, especially for the JavaScript language,
has focused on performance, to the expense of flexibility. Notably, it has
hindered our understanding of the dynamic behavior of real-world programs by
making instrumentation of existing VMs laborious and complex. It has made the
gathering of empirical data a one-time effort, by forcing researchers to patch
the source code of a production VM, therefore making the burden of maintaining the
instrumented VM up-to-date too demanding for any research team to shoulder.
With our current approaches, it is too costly to obtain longitudinal data on
the dynamic behavior of JavaScript programs.

Fortunately, the very same gains in performance can be used to regain
flexibility, even on a rigid production VM. This dissertation shows that a
sophisticated run-time optimizer can be harnessed to provide flexible run-time
instrumentation of the object model and function-calling protocol at a
performance competitive with a state-of-the-art interpreter, without having to
modify the VM source code. Our approach consists in building a meta-circular VM
targeting the source language, based on a message-sending object model. To
demonstrate the approach, we provide a reference VM for an existing programming
language, JavaScript, we show the possibility of instrumenting the object model
operations and function calls and we finally compare the baseline performance
(without instrumentation) to both a state-of-the-art interpreter and VM,
respectively SpiderMonkey without its Just-In-Time compiler and V8.

Although meta-circular VM implementations have been common in LISP languages,
as far as we know, this is the first time the approach is shown to be practical,
both in terms of performance and programming effort, on a mainstream language.
As such, this dissertation contains two original contributions:
\begin{itemize}
    \item An analysis of the required language support for our approach.     
    \item An object representation exploiting the underlying VM inline caches
        and dynamic object model to provide performance for virtualized
        operations
\end{itemize}

In addition to supporting the thesis above, we hope (1) the analysis will allow
the interested reader to evaluate if the approach could be feasible in its
favorite language, (2) to encourage future language designs to provide features
facilitating a layered approach to VM implementations and (3) the object
representation will be used by other language implementations seeking
efficiency while targeting existing JS VMs.

\section{Flexibility}

Defining and evaluating flexibility is no easy task. The usage of the term
above appeal to the intuitive notion of a system easy to tailor to one's
particular usage by modifying, removing or replacing its components. Easy might
mean that there is few lines of code to modify to use a preconfigured option,
that there is few manipulations to perform in a user interface or that unplanned
extensions could be developed in a limited time.

While intuitive, this definition is not sufficiently objective to compare
different systems. It is tied to a subjective interpretation of efforts required to
perform modifications. For the remainder of this dissertation, I'll use a more
restricted definition of flexibility. A system will be considered flexible if
it exhibits the four following properties:

\begin{itemize}
    \item \textit{Open}. Its components' behavior can be modified by
        first-class data structures and the range of possible behavior is
        maximal. 

    \item \textit{Extensible}. Its components can be independently modified or
        replaced and they support incremental definitions. 

    \item \textit{Dynamic}. It can be modified at run time.

    \item \textit{Performant}. It is fast enough to provide instant feedback on
        user modifications.
\end{itemize}

There is a continuum of support for these four properties. For example, some
parts of a system might be open but not others.  I will not try to estimate the
relative contribution of each of these properties to the overall flexibility of
a system. I will instead evaluate systems on use cases for each of these
property individually. A system could be said to be more open than another one
but not more flexible unless it is clearly equally or more open, extensible,
dynamic and performant than another one.

\section{Virtual machine}

% definition
A \textit{virtual machine} is a program that simulates the behavior of a
machine, that may or may not have a physical implementation. It may be
identical to the physical machine on which it is executing, as is the case with
current commercial solutions used to execute different operating systems as user
processes, or it might be completely different, as is the case when executing a
high-level language such as JavaScript. In this latter case, the simulated
machine might exhibit properties that do not have physical equivalents.
For example, the illusion of infinite allocatable memory can be provided by
implementing a garbage collector algorithm.~\footnote{Note, that the available
working memory is still finite and limited by the underlying physical memory
available.} 

% relationship to an operating system
In theory, there is no significant difference between a VM and
an operating system.~\footnote{"An operating system is a collection of
things that don't fit into a language. There shouldn't be one." -- Dan
Ingalls~\cite{Ingalls1981}} They both act as a middle layer between the
underlying machine and programs. They both provide abstractions and
services common to all programs. In fact, VMs for high-level
languages have been made to execute directly on hardware. In practice, the
management of hardware peripherals, memory, processing units and network
interfaces has been associated with operating systems and the support for
higher-level features of programming languages has been associated with
dedicated VMs. This historical distinction has allowed a plethora
of languages to be available for application programmers by allowing language
implementers to focus on supporting the semantic of programming languages
instead of managing the physical machine resources.

% efficiency gap
The major drawback of simulating a computer is the important efficiency
difference between a physical and a virtual implementation, the latter possibly
being orders of magnitude slower then the former. Research on VM
implementation has contributed techniques to minimize this efficiency gap.
Various high-level languages, such as Java, JavaScript, Smalltalk, Scheme or
Prolog, can now execute efficiently on general purpose processors making VMs
practical for application development.

% scope
In this dissertation, I will focus on VMs constructed to support programming
languages on top of existing operating systems or VMs. I will ignore the
virtualization of operating systems or the management of physical resources.

VMs can be described by three languages:
\begin{itemize}
    \item \textit{Source language}. Language used by programs executing on the VM.
    \item \textit{Implementation language}. Language that describe the behavior of the VM.
    \item \textit{Target language}. Language of the machine on which the VM is executing.
\end{itemize}
These languages can be different. For example, current commercial VMs for
JavaScript (source language) are written in C++ (implementation language) and
execute on Intel X86 processors (target language). Special cases, which exhibit
noteworthy properties, exist when some or all of these languages are the same. 

When its source language and its implementation language are the same, a VM is
said to be \textit{meta-circular}. Advantages of meta-circularity are resource
sharing and uniformity of the runtime. For example, a memory manager used to
manage the internal data structures of a VM can be shared with the source
language runtime. Uniformity allows optimizations written for the source
language to also apply to the implementation language. In the context of an
open system, source language code can replace implementation language code to
modify the behavior of the VM. In the litterature, \textit{self-hosting} is
also used to describe meta-circular systems. I'll restrict this latter term for
describing a system that is \textit{actually} used to produce new versions of
itself.  Therefore, a meta-circular VM needing special extensions or
performance that it cannot provide still depends on an external machine to
produce a different version of itself and is not self-hosting. Self-hosting is
therefore more strict.  Self-hosting enables a faster evolution of the system
by allowing functionalities developed for the source language to apply to the
implementation language. It frees a system from the limitations of other
available systems that would be needed otherwise.

When its source language and its target language are the same, we usually
describe a VM as performing \textit{source-to-source} translation. This is
especially useful to provide additional or different properties to the
executing program while avoiding the implementation of complicated
functionalities, such as floating-point or regular expression support.  It can
be used to reduce a complicated language to a kernel language using only a
subset of all the functionalities offered in the source language. This is
especially useful when implementing mainstream languages possessing redundant
features acquired through backward-compatibility-constrained evolution. It may
ease reasoning about the program, facilitate instrumentation or exploit
optimizations written for some functionalities to accelerate others.

When its implementation language and its target language are the same, an
\textit{executable} VM can execute on a machine without requiring any external
support. Few VMs are implemented directly in a target low-level language unless
an executable VM is needed but no translator from a high-level language to the
target low-level language are available. When translators are available,
executable VMs are usually the result of an automatic translation from a high-level
description.

When all languages are the same, I will say a VM is \textit{platonic}.  I would
argue that a platonic VM is possibly the simplest that could be built for a
given language. Only the elements of interest in the language can be acted
upon, the others can be left as is without having to provide an implementation
in a lower-level language.  Tools supporting the source language can be used to
inspect, modify and debug the VM, creating a highly productive environment.
Optimizations for the new functionalities can target existing optimisations of
the source language to efficiently implement new functionalities and properties
without having to specify low-level mecanims that would otherwise be required.
It greatly simplifies exploration of novel ideas. More importantly, it suggests
a metric to evaluate the cost of providing fonctionalities or properties that
are not available in the original source language: the overhead of executing an
original source program on the VM compared to bare execution on the reference
machine. When the overhead is small, it is an indication that the functionality
\textit{can} be implemented efficiently using known techniques. However, if the
overhead is high, the functionality \textit{may} be implemented using known
techniques or techniques yet to be invented.

In this dissertation, using a platonic JS VM, I show that the object model
presented is practical performance-wise by showing that is has a low
overhead on given benchmarks for every object operation it implements. 

\section{JavaScript}

JS is a dynamic language, imperative but with a strong functional component,
and a prototype-based object system similar to that of Self.

A JS object contains a set of properties (a.k.a. fields in other OO languages),
and a link to a parent object, known as the object's prototype. Properties are
accessed with the notation \kw{obj.prop}, or equivalently \kw{obj["prop"]}.
This allows objects to be treated as dictionaries whose keys are strings, or as
one dimensional arrays (a numeric index is automatically converted to a
string).  When fetching a property that is not contained in the object, the
property is searched in the object's prototype recursively. When storing a
property that is not found in the object, the property is added to the object,
even if it exists in the object's prototype chain. Properties can also be
removed from an object using the \kw{delete} operator. JS treats global
variables, including the top-level functions, as properties of the global
object, which is a normal JS object.

Anonymous functions and nested functions are supported by JS. Function objects
are closures which capture the variables of the enclosing functions. Common
higher-order functions are predefined. For example, the \kw{map}, \kw{forEach}
and \kw{filter} functions allow processing the elements of an array of data
using closures, similarly to other functional languages. All functions accept
any number of actual parameters. The actual parameters are passed in an array,
which is accessible as the \kw{arguments} local variable. The formal parameters
in the function declaration are nothing more than aliases to the corresponding
elements at the beginning of the array. Formal parameters with no corresponding
element in the array are bound to a specific undefined value.

JS also has reflective capabilities (enumerating the properties of an object,
testing the existence of a property, etc.) and dynamic code execution
(\kw{eval} of a string of JS code).  The next version of the standard is
expected to add proper tail calls, rest parameters, block-scoped variables,
modules, and many other features.

Initially, our research initiative chose JS as a research object because it is
widely used to write web applications and there is a performance gap between
benchmarks executing on the fastest implementations and compiled with C/C++
compilers, suggesting there are techniques to be found to close the gap.
During exploration of VM implementation techniques, the dynamic nature of the
language was found to be particularly well-suited to base a flexible design
around it and initiated the work presented here.

\section{Object Model}

An object model is a set of object kinds, their supported operations and the
time at which those operations can be performed.  Examples of object kinds are
arrays, numbers, associative arrays and classes.  Examples of operations are
object creation, addition, removal or update of properties as well as
modification of the inherintance chain. Examples of time for performing
operations are \textit{run time}, when a program is executing, \textit{compile
time}, when a program is being compiled or \textit{edit time}, when a program's
source code is being modified. An object model structures programs to obtain
properties on the resulting system such as security, extensibility or
performance. Most object models for programming languages provide an
inheritance mecanism to facilitate \textit{extensibility} by allowing an object
to be defined incrementally defined in terms of an existing object.

Different languages have different object models. \textit{Class-based}
object-oriented languages use \textit{class} objects to describe the properties
and the inheritance chain of \textit{instance} objects. For example, C++ class
objects exist only at compile time. Property values can be updated at run time.
Properties can only be added or removed at edit time and all their accesses are
verified at compile time. Java use run-time objects to represent classes and
allow new classes to be added at run time.  Classes cannot be modified at run
time unless their new definition is reloaded through the debugger API. Ruby
class objects exist at run time and new properties can be added also at run
time. \textit{Prototype-based} object-oriented languages forego the difference
between class and instance objects. Objects and their inheritance chain is
defined directly on objects. Self and JavaScript object properties can be
added or removed at run time. 

\textit{Message sending} is an operation that invoke a behavior on an object by
executing a program associated to a given message.  A message can exist at run
time and be an object like any other or it can exist only at compile time.
Message sending \textit{decouples} the intention of a program from its
implementation by adding a level of indirection between the invocation and the
execution of a given behavior. This indirection allows the behavior to change
during a program's execution. Therefore, it is a source of \textit{dynamism}.

Some languages call the program associated to a message, a \textit{method
implementation}, the message, a \textit{method name} and the act of sending a
message, \textit{method calling}. This terminology is closely tied to an
implementation strategy where the implementation is a function, the method name
is a compile-time symbol and the method call is a lookup followed by a
synchronous function call in the same address space. I will prefer the
message-sending terminology because it is more abstract and would easily
accomodate an asynchronous calling mecanism that could happen between different
address spaces (i.e. machines). 

A message-sending object model is an object model that takes message sending as
its primitive operation. It defines every other operation in terms of message
sends. By doing so, every other operation of the object model becomes dynamic,
i.e. it can change at run time.

An object model defined in terms of itself is an object model where every
component required to implement the object model operations are themselves
objects. For example, the behavior of objects might be functions, the messages
might be symbols and classes might associate symbols to functions. By
recursively defining the object model in terms of itself, it becomes
\textit{open} to modification from regular object operations.

A message-sending object model defined in terms of itself combines both the
dynamism of message sending with the openness of recursivity. By providing
every operation through a message send and having every component part of the
object model, it effectively encapsulate its implementation strategy,
facilitating its evolution. By being based on a message sending primitive, the
optimization effort can be focused on this primitive and run-time information
can be used to specialize the behavior invoked, providing an opportunity for
\textit{performance}. 

The next pages show how a message-sending object model for JS defined in terms of
itself provides \textit{openness}, \textit{dynamism}, \textit{extensibility}
and \textit{performance} and empirically evaluate its practicality for building
\textit{flexible} virtual machines.

\section{Outline}
\begin{itemize}
    \item The next chapter presents previous work on flexible computer systems.
    \item Chapter~\ref{chap:Design} explains the design of the object model
    independently of the implementation or the target language of the VM.
    \item Chapter~\ref{chap:Implementation} explains implementation
    peculiarities for a particular target and implementation language, JS.
    \item Chapter~\ref{chap:Optimizations} describes the optimization
    techniques used to remove much of the dynamic overhead of the design,
    tailored to a particular JS implementation.  
    \item Chapter~\ref{chap:Flexibility} presents use cases in modifying the VM
    that are either difficult or impossible to do with existing JS
    implementations.
    \item Chapter~\ref{chap:Complexity} evaluates the complexity of the
    resulting system using the number of source lines of code as a proxy.
    \item Chapter~\ref{chap:Performance} compares my implementation with a
    state-of-the-art implementation to establish the overhead of providing the
    flexibility.  
    \item The conclusion presents the lessons I learned in building, whole or
    part, three VMs during the course of this research project. I conclude with
    recommandations on browser behavior that would allow this design to
    serve as a basis for web pages to provide their own runtime, relaxing
    current hard-constraints on web application evolution.
\end{itemize}
